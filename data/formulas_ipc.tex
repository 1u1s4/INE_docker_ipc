{\setlength{\parindent}{0cm} 
\newcommand{\IPC}[1]{\text{IPC}_{#1}}
\textbf{Cálculo de Índices de producto}\\
Para realizar el cálculo del índice de cada producto se utiliza la ecuación
\begin{align*}
	I_{k}^{t} =  \prod \left( \frac{p_{i}^{t}}{p_{i}^{t-1}} \right)^{\frac{1}{n}} \times I_{k}^{t-1}
\end{align*}
donde
\begin{itemize}
    \item[]$I_{k}^{t}$ = Índice actual del producto $k$ del período $t$
    \item[]$p_{i}^{t}$ = Precio de la variedad $i$ del período $t$ (actual)
    \item[]$p_{i}^{t-1}$ = Precio de la variedad $i$ del período $t-1$ (anterior)
    \item[]$I_{k}^{t-1}$ = Ínidce del producto k del período $t-1$ (anterior)
\end{itemize}

\textbf{Cálculo de Índices agregados}\\
Para realizar el cálculo de los índices de gregados (niveles superiores) se utiliza la ecuación
\begin{align*}
	I_{j}^{t} = \frac{ \sum_{k=1}^{k} w_{k} \times I_{k}^{t}}{w_{j}}
\end{align*}
donde

$I_{j}^{t}$ = Índice actual del agregado $j$ del periodo $t$\\
$k$ = número de productos que integran el agregado $j$\\
$w_{k}$ = Ponderación del producto $k$\\
$ I_{k}^{t}$ = índice actual del producto $k$ del período $t$ (actual)\\
$w_{j}$ = Sumatoria de $w_{k}$ (Ponderación del agregado $j$)\\

\textbf{Variación Intermensual}\\
Se obtiene relacionando el Índice de Precios al Consumidor del mes de estudio $\IPC{t}$ con el correspondiente al mes anterior $\IPC{t-1}$, mediante la fórmula siguiente:
\begin{align*}
	\text{Variación intermensual} = \left( \frac{\IPC{t}}{\IPC{t-1}} \right) \times 100
\end{align*}

\textbf{Variación interanual}\\
Se obtiene relacionando el Índice de Precios al Consumidor del mes de estudio $\IPC{t}$ con el del mismo mes del año anterior $\IPC{t-12}$, mediante la fórmula siguiente:
\begin{align*}
	\text{Variación intermensual} = \left( \frac{\IPC{t}}{\IPC{t-12}} \right) \times 100
\end{align*}

\textbf{Incidencia o impacto}\\
La incidencia o impacto del gasto básico, división de gasto o región $x$, en el mes actual, se obtiene mediante la fórmula:
\begin{align*}
	\Delta_{t}^{x} = w_x \left( \frac{I_t^x - I_{t-1}^x}{\IPC{t-1}} \right)
\end{align*}
donde $I_t^x$ es el número índice del gasto básico, división de gasto o región $x$ en el periodo $t$, $w_x$ es la ponderación del gasto básico, división de gasto o región $x$ dentro de la canasta familiar de consumo. $\IPC{t-1}$ es el Índice de Precios al Consumidor del mes anterior.\\

\textbf{Poder adquisitivo de la moneda}\\
El poder adquisitivo de la moneda en el mes de estudio respecto del mes base, se obtiene mediante la fórmula:
\begin{align*}
	\text{PA}_t = \frac{1}{\IPC{t}} \times 100
\end{align*}
}
